\section*{Glossaire}\label{sec:glossaire}

\begin{description}
    \item [Surpoids]\label{itm:surpoids}
    {
        La population en surpoids ou obèse est la part de la population ayant un poids excessif présentant des risques
        pour la santé en raison d'une proportion élevée de tissu adipeux.
        L'outil de mesure le plus fréquemment utilisé est l'indice de masse corporelle (IMC), qui évalue le poids d'un
        individu par rapport à sa taille ($\frac{\text{poids}}{\text{taille}^2}$,
        le poids étant exprimé en kilogrammes et la taille en mètres).
        Selon la classification de l'OMS, les adultes présentant un IMC compris entre 25 et 30 sont en surpoids, et ceux
        dont l'IMC est égal ou supérieur à 30 sont considérés comme obèses.
        Cet indicateur est calculé à partir de données autodéclarées (poids et taille fournis lors d'enquêtes) et de
        données mesurées (estimations précises tirées d'examens médicaux).
        Il est exprimé en pourcentage de la population âgée de 15 ans et plus.
    }
\end{description}